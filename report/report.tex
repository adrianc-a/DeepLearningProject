\documentclass[conference]{IEEEtran}
\usepackage{cite}

\ifCLASSINFOpdf
  % \usepackage[pdftex]{graphicx}
  % declare the path(s) where your graphic files are
  % \graphicspath{{../pdf/}{../jpeg/}}
  % and their extensions so you won't have to specify these with
  % every instance of \includegraphics
  % \DeclareGraphicsExtensions{.pdf,.jpeg,.png}
\else
  % or other class option (dvipsone, dvipdf, if not using dvips). graphicx
  % will default to the driver specified in the system graphics.cfg if no
  % driver is specified.
  % \usepackage[dvips]{graphicx}
  % declare the path(s) where your graphic files are
  % \graphicspath{{../eps/}}
  % and their extensions so you won't have to specify these with
  % every instance of \includegraphics
  % \DeclareGraphicsExtensions{.eps}
\fi
\usepackage{amsmath}

% correct bad hyphenation here
\hyphenation{op-tical net-works semi-conduc-tor}


\begin{document}
% paper title
\title{AlphaGo Zero Techniques on other games}

\author{
Adrian Chmielewski-Anders, Parsoa Khorsand, Sadegh Shams \\
\{achmielewski, kanalbant, pkhorsand, mshamsabardeh\}@ucdavis.edu
}

\maketitle

\begin{abstract}
    Our abstract
\end{abstract}


% For peer review papers, you can put extra information on the cover
% page as needed:
% \ifCLASSOPTIONpeerreview
% \begin{center} \bfseries EDICS Category: 3-BBND \end{center}
% \fi
%
% For peerreview papers, this IEEEtran command inserts a page break and
% creates the second title. It will be ignored for other modes.
%\IEEEpeerreviewmaketitle



\section{Introduction}
\begin{enumerate}
    \item
    Intro part 1
    \item
    Intro part 2

\end{enumerate}

\section{Methods}
    Our methods \cite{AlphaGoZero}.
\section{Results}
    Our results.

\section{Discussion}
    Discuss.

\section{Conclusion}
    Concluding section.


\bibliography{references}
\bibliographystyle{IEEEtran}

\section{Author Contributions}
    Contributions part.
\end{document}
